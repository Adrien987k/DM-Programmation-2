	\documentclass{article}

	\usepackage[utf8x]{inputenc} 	% accents
	\usepackage[T1]{fontenc}      % caractères français
	\usepackage{geometry}         % marges
	\usepackage[french]{babel}  	% langue
	\usepackage{graphicx}         % images
	\usepackage{verbatim}         		% texte préformaté
	\usepackage{bussproofs}       		% proofs
	\usepackage[cache=false]{minted}         % source code

	\title{DM de programmation 2}
	\author{Adrien Bardes}
	\date{30 mars 2018}

	\begin{document}

	\maketitle

	\section{La spécification de Mini-Ml}

	\paragraph{Question 1.}

	\begin{prooftree}
	\def\fCenter{ \vdash\ }
	\Axiom$\Gamma \fCenter n : int$
	\UnaryInf$\Gamma \fCenter succ n : int$
	\end{prooftree}

	\begin{prooftree}
	\def\fCenter{ \vdash\ }
	\Axiom$\Gamma \fCenter n : int$
	\UnaryInf$\Gamma \fCenter pred n : int$
	\end{prooftree}


	\paragraph{Question 2.}

	\begin{minted}{ocaml}

	let plus_body = fun n -> fun m -> 
		let n = fst c in
		let m = snd c in
		if n = 0 then n2 else
		if m = 0 then n1 else
		plus (pred n, succ m)

	let rec plus = plus_body in plus


	let time_body = fun n -> fun m ->
		if n = 0 then 0 else
		if m = 0 then 0 else
		time (plus(n, m), pred m)

	let rec time = time_body in time

	\end{minted}

	\paragraph{Question 3.}

	\begin{minted}{ocaml}

	let rec f = fun x ->
		if x = 0 then 1 else
		time (x, f (x - 1))

	\end{minted}

	%TODO arbre de dérivation

	\paragraph{Question 4.}

	\begin{minted}{ocaml}

	let x = fst ((fun n -> fun m -> (n, m)) 2 3) in x

	\end{minted}

	\paragraph{Question 5.}

	Le typage des expression assure l'absence de plantage du à un mauvais typage
	lors de l'exécution d'un programme

	\paragraph{Question 6.}

	En Ocaml, toutes les expressions sont des valeurs. Par exemple dans le code suivant, f est une valeur :
	\begin{minted}{ocaml}
	let rec f x = if x = 0 then 0 else f (x -1)
	\end{minted}

	\paragraph{Question 7.}

	Sans les opérateurs de comparaison $\le$ et $\ge$ il n'est pas possible d'implémenter $pred$ en Mini-Ml.

	\paragraph{Question 8.}



	\paragraph{Question 9.}

	L'opérateur $\not\equiv$ effectue un test sur les types des deux élements comparés. On a $a \not\equiv b$ si et seulement si $a$ et $b$ sont de type différents ou $a$ et $b$ sont du même type mais pas la même valeur.


	\section{Implémentation de Mini-ML}

	\paragraph{Question 10.}

	%Let of var * ty option * expr loc
	%let x^A = u
	%let rec x^A = u
	%LetRec of var * ty option * expr loc

	\def\fCenter{ \mbox{ $\vdash$\ }}

	\begin{figure}[!t]
	\centering
	\caption{La commande Let}
	\label{Tux}
	\end{figure}

	\begin{prooftree}
	\AxiomC{$\Gamma \vdash t : A$}
	\UnaryInfC{$\Gamma \fCenter \textbf{let}\ x^A = t$}
	\end{prooftree}

	\begin{prooftree}
	\AxiomC{$\Gamma \vdash \textbf{let}\ x = \textbf{fix}\ (\lambda x^A .t)$}
	\UnaryInfC{$\Gamma \vdash \textbf{let\ rec}\ x^A = t$}
	\end{prooftree}

	\begin{prooftree}
	\AxiomC{$t \hookrightarrow t'$}
	\UnaryInfC{$\textbf{let}\ x = t\ \hookrightarrow \textbf{let}\ x = t'$}
	\end{prooftree}

	\begin{prooftree}
	\AxiomC{$$}
	\UnaryInfC{$\textbf{let\ rec}\ x^A = t \hookrightarrow \textbf{fix}\ (\lambda x^A .t)$}
	\end{prooftree}

	\paragraph{Question 11.}

	Un programme Mini-ML est une suite de commandes.

	\paragraph{Question 12.}

	$\Gamma, let x = t$ ==> $\Gamma, let x = t in x$
	$\Gamma, let rec x = t$ ==> $let rec x = t in x$

	\paragraph{Question 13.}

	\begin{prooftree}
	\AxiomC{$\Gamma, x : \textbf{bool}$}
	\AxiomC{$\Gamma, y : \textbf{bool}$}
	\BinaryInfC{$\Gamma, x\ \&\& \ y : \textbf{bool}$}
	\end{prooftree}

	\begin{prooftree}
	\AxiomC{$x \hookrightarrow \textbf{true}$}
	\AxiomC{$y \hookrightarrow \textbf{true}$}
	\BinaryInfC{$x\ \&\& \ y \hookrightarrow \textbf{true}$}
	\end{prooftree}

	\begin{prooftree}
	\AxiomC{$x \hookrightarrow \textbf{false}$}
	\UnaryInfC{$x\ \&\& \ y \hookrightarrow \textbf{false}$}
	\end{prooftree}

	\begin{prooftree}
	\AxiomC{$y \hookrightarrow \textbf{false}$}
	\UnaryInfC{$x\ \&\& \ y \hookrightarrow \textbf{false}$}
	\end{prooftree}

	\begin{prooftree}
	\AxiomC{$\Gamma, x : \textbf{nat}$}
	\AxiomC{$\Gamma, y : \textbf{nat}$}
	\BinaryInfC{$\Gamma, x + y : \textbf{nat}$}
	\end{prooftree}

	\begin{prooftree}
	\AxiomC{$\textbf{plus}\ x \ y$}
	\UnaryInfC{$x + y$}
	\end{prooftree}

	\paragraph{Question 14.}

	Le let rec in est directement traduit en un fixpoint...

	\end{document}
