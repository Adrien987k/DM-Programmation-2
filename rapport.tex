\documentclass[12pt]{article}

\usepackage[utf8x]{inputenc} 	% accents
\usepackage[T1]{fontenc}      % caractères français
\usepackage{geometry}         % marges
\usepackage[french]{babel}  	% langue
\usepackage{graphicx}         % images
\usepackage{verbatim}         		% texte préformaté
\usepackage{bussproofs}       		% proofs
\usepackage[cache=false]{minted}         % source code


\geometry{hmargin=2.5cm,vmargin=1.5cm}

\newenvironment{bprooftree}
  {\leavevmode\hbox\bgroup}
  {\DisplayProof\egroup}


\title{DM de programmation 2}
\author{Adrien Bardes}
\date{30 mars 2018}

\begin{document}

\maketitle

\section{La spécification de Mini-Ml}

\paragraph{Question 1.}

\[
\begin{bprooftree}
\AxiomC{$\Gamma \vdash\ n : int$}
\UnaryInfC{$\Gamma \vdash\ \textbf{succ} \ n : int$}
\end{bprooftree} \qquad
\begin{bprooftree}
\AxiomC{$t \hookrightarrow t^{\prime}$}
\UnaryInfC{$\textbf{succ} \ t \hookrightarrow \textbf{succ} \ t^{\prime}$}
\end{bprooftree} \qquad
\begin{bprooftree}
\AxiomC{$n \ \textbf{value}$}
\UnaryInfC{$\textbf{succ} \ n \hookrightarrow (n + 1) \ \textbf{value}$}
\end{bprooftree}
\]

\[
\begin{bprooftree}
\AxiomC{$\Gamma \vdash\ n : int$}
\UnaryInfC{$\Gamma \vdash\ \textbf{pred} \ n : int$}
\end{bprooftree} \qquad
\begin{bprooftree}
\AxiomC{$t \hookrightarrow t^{\prime}$}
\UnaryInfC{$\textbf{pred} \ t \hookrightarrow \textbf{pred} \ t^{\prime}$}
\end{bprooftree}
\]

\[
\begin{bprooftree}
\AxiomC{$n \ \textbf{value}$}
\AxiomC{$n = (0 \ \textbf{value})$}
\BinaryInfC{$\textbf{pred} \ n \hookrightarrow 0 \ \textbf{value}$}
\end{bprooftree} \qquad
\begin{bprooftree}
\AxiomC{$n \ \textbf{value}$}
\UnaryInfC{$\textbf{pred} \ n \hookrightarrow (n - 1) \ \textbf{value}$}
\end{bprooftree}
\]

\paragraph{Question 2.}

\begin{minted}{ocaml}
let rec plus = fun n -> fun m ->
	if n = 0 then m else
	if m = 0 then n else
	plus (pred n) (succ m)
in plus

let rec time = fun n -> fun m ->
	if n = 0 then 0 else
	if m = 0 then 0 else
	if n = 1 then m else
	if m = 1 then n else
	time (plus n m) (pred m)
in time
\end{minted}

\paragraph{Question 3.}

\begin{minted}{ocaml}
let rec f = fun x ->
	if x = 0 then 1 else
	time x (f (x - 1))
	in f
\end{minted}

%TODO arbre de dérivation

\paragraph{Question 4.}

\begin{minted}{ocaml}
let x = fst (2,3) in
let y = snd (2,3) in
x
\end{minted}

\paragraph{Question 5.}

Le typage des expressions assure que le programme s'évaluera en une valeur, en suivant les règles de la sémantique. Un mauvais typage donne lieu à des expressions sur lesquelles aucune règle de la sémantique ne s'applique. Par example si dans l'expression : $u \ v$, $u$
n'est du type $\lambda x^{A}.t \ \textbf{value}$, l'expression n'a aucun sens, ce n'est pas une valeur et aucune règle de la sémantique ne s'applique.


\paragraph{Question 7.}

Sans les opérateurs de comparaison $\le$ et $\ge$ il n'est pas possible d'implémenter $pred$ en Mini-Ml. En effet, on ne dispose d'aucune relation d'ordre sur les $n \ \textbf{value}$.


\paragraph{Question 8.}

Dans le code suivant, dans g\_body et f\_body, on peut faire appel à l'expression f ou g.

\begin{minted}{ocaml}
let rec f = fun x ->
	let rec g = fun y ->
		g_body
	in
	f_body
in
f
\end{minted}


\paragraph{Question 9.}

L'opérateur $\not\equiv$ effectue un test sur les types des deux élements comparés. On a $a \not\equiv b$ si et seulement si $a$ et $b$ sont de type différents ou $a$ et $b$ sont du même type mais pas la même valeur. Il est possible de faire sans, en interdisant les comparaisons entre deux valeurs de type différents. Ainsi un simple $\neq$ suffit.


\section{Implémentation de Mini-ML}


\paragraph{Question 11.}

Il est possible de se passer des commandes. En effet, on peut remplacer
\begin{minted}{ocaml}
let rec f = ... par let rec f = ... in f 
et
let f = ... par let f = ... in f
\end{minted}


\paragraph{Question 12.}

On utilise la transformation de la question 11.


\paragraph{Question 13.}

\[
\begin{bprooftree}
\AxiomC{$\Gamma, x : \textbf{bool}$}
\AxiomC{$\Gamma, y : \textbf{bool}$}
\BinaryInfC{$\Gamma, x\ \&\& \ y : \textbf{bool}$}
\end{bprooftree} \qquad
\begin{bprooftree}
\AxiomC{$x \hookrightarrow \textbf{true}$}
\AxiomC{$y \hookrightarrow \textbf{true}$}
\BinaryInfC{$x\ \&\& \ y \hookrightarrow \textbf{true}$}
\end{bprooftree} \qquad
\begin{bprooftree}
\AxiomC{$x \hookrightarrow \textbf{false}$}
\UnaryInfC{$x\ \&\& \ y \hookrightarrow \textbf{false}$}
\end{bprooftree}
\]

\[
\begin{bprooftree}
\AxiomC{$y \hookrightarrow \textbf{false}$}
\UnaryInfC{$x\ \&\& \ y \hookrightarrow \textbf{false}$}
\end{bprooftree} \qquad
\begin{bprooftree}
\AxiomC{$\Gamma, x : \textbf{nat}$}
\AxiomC{$\Gamma, y : \textbf{nat}$}
\BinaryInfC{$\Gamma, x + y : \textbf{nat}$}
\end{bprooftree} \qquad
\begin{bprooftree}
\AxiomC{$\textbf{plus}\ x \ y$}
\UnaryInfC{$x + y$}
\end{bprooftree}
\]

\[
\begin{bprooftree}
\AxiomC{$t \hookrightarrow t^{\prime}$}
\AxiomC{$u \hookrightarrow u^{\prime}$}
\BinaryInfC{$t \ \&\& \ u \hookrightarrow t^{\prime} \ \&\& \ u^{\prime}$}
\end{bprooftree} \qquad
\begin{bprooftree}
\AxiomC{$t \hookrightarrow t^{\prime}$}
\AxiomC{$u \hookrightarrow u^{\prime}$}
\BinaryInfC{$t + u \hookrightarrow t^{\prime} + u^{\prime}$}
\end{bprooftree}
\]

\paragraph{Question 14.}

L'expression 
\begin{minted}{ocaml}
let rec f in ...
\end{minted} 
est directement traduite dans l'AST par l'expresison 
\begin{minted}{ocaml}
fix(let f in ...)
\end{minted}
$\textbf{fix}$ est un opérateur de point fixe.


\paragraph{Question 15.}

Avant typage, le type d'une abstraction est None. Cependant on laisse la possibilité d'écrire
\begin{minted}{ocaml}
let f = fun (x:t) -> ...
\end{minted}
Auquel cas l'abstraction est déjà typée et a le type $Some(t)$.
Après typage, si celui-ci réussi, toutes les abstractions ont un type de la forme $Some(t)$.

Lors du typage, on procède à l'unification de termes. Si une lambda a pour type $Some(t)$ et que le typeur détermine que $t = int \rightarrow int$, ont aurait envie de modifier dans toutes les occurences de la lambda, $Some(t)$ par $Some(int \rightarrow int)$. Cela se fait très simplement en utilisant une référence.


\paragraph{Question 20.}

En C, toute fonction est close. Un programme Mini-ML comporte des fonctions non closes. La \textit{closure conversion}, qui transforme ces fonctions non closes en fonction close est donc nécessaire.

La \textit{defunctionalization} permet de supprimer les fonctions d'ordre supérieur. En C ces fonctions difficilement manipulables, cela est donc très utile.



\paragraph{Question 27.}

Si le typage échoue, le numéro de la ligne et du caractère où se trouve l'erreur sont affichés.


\end{document}
