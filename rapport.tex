

\documentclass[a4paper, titlepage]{livret}

\usepackage[latin1]{inputenc} % accents
\usepackage[T1]{fontenc}      % caractères français
\usepackage{geometry}         % marges
\usepackage[francais]{babel}  % langue
\usepackage{graphicx}         % images
\usepackage{verbatim}         % texte préformaté
\usepackage{bussproofs}       % proofs
\usepackage{minted}         % source code

\title{DM de programmation 2}
\author{Adrien Bardes}
\date{30 mars 2018}

\pagestyle{headings}          % affiche un rappel discret (en haut à gauche)
                              % de la partie dans laquel on se situe


\begin{document}

\section{La spécification de Mini-Ml}

\paragraph{Question 1.}

\begin{mathpar}

\begin{prooftree}
\def\fCenter{ \vdash\ }
\Axiom$\gamma \fCenter n : int$
\UnaryInf$\gamma \fCenter succ n : int$
\end{prooftree}

\begin{prooftree}
\def\fCenter{ \vdash\ }
\Axiom$\gamma \fCenter n : int$
\UnaryInf$\gamma \fCenter pred n : int$
\end{prooftree}

\end{mathpar}

\paragraph{Question 2.}

\begin{minted}{ocaml}

let plus_body = fun n -> fun m -> 
  let n = fst c in
  let m = snd c in
  if n = 0 then n2 else
  if m = 0 then n1 else
  plus (pred n, succ m)

let rec plus = plus_body in plus


let time_body = fun n -> fun m ->
  if n = 0 then 0 else
  if m = 0 then 0 else
  time (plus(n, m), pred m)

let rec time = time_body in time

\end{minted}

\paragraph{Question 3.}

\begin{minted}{ocaml}

let rec f = fun x ->
  if x = 0 then 1 else
  time (x, f (x - 1))

\end{minted}

%TODO arbre de dérivation

\paragraph{Question 4.}

\begin{minted}{ocaml}

let x = fst ((fun n -> fun m -> (n, m)) 2 3) in x

\end{minted}

\paragraph{Question 5.}

Le typage des expression assure l'absence de plantage du à un mauvais typage
lors de l'exécution d'un programme

\paragraph{Question 6.}

\begin{itemize}

\item float
\item n-uplet
\item bool pour true et false
\item ...

\end{itemize}

\paragraph{Question 7.}




\section{Implémentation de Mini-ML}



\end{document}
